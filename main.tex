\documentclass[12pt]{article}
\usepackage[spanish]{babel}
\usepackage{amsmath}
\usepackage{amssymb}
\usepackage{geometry}
\geometry{a4paper, margin=2cm}
\usepackage{booktabs}
\usepackage{setspace}

\title{Hoja de referencia}
\author{Samuel Reyna}

\begin{document}
\maketitle
\section{Prestaciones}

\subsection*{{Definición}}

Cuando se afirma que un computador tiene mejores prestaciones que otro, 
se hace referencia a la rapidez con la que completa una tarea. 
La definición de "prestaciones" puede ser sutil, como se ilustra con la analogía de los aviones, 
donde la "mejor" prestación puede variar si se mide por velocidad, autonomía
 o productividad de pasajeros (capacidad multiplicada por velocidad de crucero).

\subsection*{Tiempo de ejecución como métrica clave}
La única métrica totalmente válida para evaluar las prestaciones es el tiempo de ejecución. 
Se considera que una máquina tiene mejores prestaciones si puede completar una tarea en menos tiempo.

La relación entre las prestaciones de dos máquinas (X e Y) se puede expresar con la siguiente fórmula:

\subsection*{Fórmula}
\begin{itemize}
    \item Tiempo de ejecución Y =$T_Y$
    \item Tiempo de ejecución x =$T_X$
\end{itemize}


    \begin{spacing}{3.0} 
        \begin{center}  
            $\displaystyle\frac{Prestaciones_X}{Prestaciones_Y}=\frac{T_Y}{T_X}$\\
        \end{center}
    \end{spacing}

Así, si las prestaciones de una máquina X son mayores que las prestaciones de una máquina Y, se tiene:\\


    \begin{spacing}{3.0} % 1.5 veces el espacio normal
        \begin{center}  
            $Prestaciones_X > Prestaciones_Y$

            $\displaystyle\frac{1}{T_Y}=\frac{1}{T_X}$

            $T_X > T_Y$
        \end{center}
    \end{spacing}

    Esto significa que el tiempo de ejecución de Y es mayor que el de X, si X es más
     rápido que Y.\\
 Al tratar sobre el diseño de un computador, a menudo se desea relacionar
 cuantitativamente las prestaciones de dos máquinas diferentes. Usaremos la frase
 “X es n veces más rápida que Y” para indicar que:\\

 \begin{spacing}{3.0} 
        \begin{center}  
            $\displaystyle\frac{Prestaciones_X}{Prestaciones_Y}=n$\\
        \end{center}
    \end{spacing}

     Si X es n veces más rápida que Y, entonces el tiempo de ejecución de Y es n veces
 mayor que el de X:
 \begin{spacing}{3.0} % 1.5 veces el espacio normal
        \begin{center}  
            $\displaystyle\frac{Prestaciones_X}{Prestaciones_Y}=\frac{T_Y}{T_X}=n$\\
        \end{center}
    \end{spacing}

    \section{ Prestaciones de la CPU y sus factores}
    Frecuentemente, diseñadores y usuarios miden las prestaciones usando métricas
 diferentes. Si se pudieran relacionar estas métricas, se podría determinar el efecto
 de un cambio en el diseño sobre las prestaciones observadas por el usuario. Como
 nos estamos restringiendo a las prestaciones de la CPU, la medida base de prestaciones
  será el tiempo de ejecución de la CPU. Una fórmula sencilla que relaciona
 las métricas más básicas (ciclos de reloj y tiempo del ciclo de reloj) con el tiempo
 de CPU es la siguiente:

 \subsection*{Fórmula}
\begin{itemize}
    \item Tiempo de ejecución de CPU= $T_{CPU}$
    \item Ciclos de la CPU =$Ciclos_{CPU}$
    \item  Tiempo del ciclo del reloj = $T_{cicloR}$
    \item Frecuencia del reloj: $Frecuencia$
\end{itemize}

\begin{spacing}{3.0} % 1.5 veces el espacio normal
        \begin{center}  
            $T_{CPU} = Ciclos_{CPU} \times T_{cicloR}$\\
            $T_{CPU} = \displaystyle\frac{Ciclos_{CPU}}{Frecuencia}$
        \end{center}
    \end{spacing}
    
\subsection*{ Prestaciones de las instrucciones}
\begin{itemize}
    \item Instrucciones de un programa =$I$
    \item Media de instrucciones por ciclo =$CPI$
\end{itemize}

        \begin{center}  
            $Ciclos_{CPU} = I \times CPI$
        \end{center}

         Ciclos de reloj por instrucción (CPI): número medio de ciclos de reloj 
        por instrucción para un programa o fragmento de programa.
        
    \section{La ecuación clásica de las prestaciones de la CPU}
    Ahora se puede escribir la ecuación básica de las prestaciones en términos del
    número de instrucciones (número de instrucciones ejecutadas por el programa),
    del CPI y del tiempo de ciclo:
    
    \subsection*{Fórmula}
    \begin{itemize}
        \item Tiempo de ejecución = $T_{EXE}$
        \item Número de instrucciones = $N_{Inst}$
        \item Tiempo de ciclo = $T_{ciclo}$
    \end{itemize}

    \begin{spacing}{3.0} % 1.5 veces el espacio normal
        \begin{center}  
            $T_{EXE} = N_{Inst} \times T_{ciclo} \times CPI$ \\
            $T_{EXE} = \displaystyle\frac{I \times CPI}{Frecuencia}$
        \end{center}
    \end{spacing}

    Número de instrucciones: número de instrucciones ejecutadas por el 
    programa.

      
      \begin{center}  
            Ciclos de reloj de la CPU = $\sum_{i=1}^{n}(CPI_i \times I_i)$
        \end{center}
\end{document} 